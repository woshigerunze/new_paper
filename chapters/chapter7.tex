\chapter{总结与展望}
本章将对本文的自动驾驶运行时通信系统研究内容进行全面总结,还将针对本系统中的不足提出下一阶段的优化方向。
\section{论文工作总结}
在自动驾驶行业飞速发展的今天,面向自动驾驶的开发者开发平台孕育而生,让越来越多的开发者能够方便、快捷的开发
自动驾驶应用。小到自动驾驶某个算法大到整个自动驾驶软件架构都有完整的供应方案,自动驾驶公司在这种情况下不能忽视
自研的重要性,否则将在商业化应用时受到来自应用协议许可和供应商的压力,本系统正式在这种背景下提出的。

本文基于目前自动驾驶各算法模块间通信使用的通信系统存在的特点和可进一步优化的情况下,提出了解决方案
具体工作内容如下:
\begin{enumerate}
    \item 从自动驾驶各算法模块间通信和自动驾驶传感器数据的特点入手,结合自动驾驶系统通信业务对系统进行
    详细的需求分析。基于方便用户使用和高通信性能的设计出发点,将自动驾驶运行时通信系统分为通信单元模块、服务模块、
    通信抽象模块、通信传输模块、任务模块、调度模块、服务发现模块和中心节点共八个边界明确的模块。既契合了模块化的
    软件设计思想,也向用户屏蔽了系统底层复杂的实现。
    \item 系统的设计与实现。基于对自动驾驶运行时通信系统的需求分析,从系统架构设计、系统组网方案设计和各功能模块及其
    接口设计对系统的技术方案、重要算法流程和模块间的调用链进行详细说明和描述。在通信方式选型上,本文选用了
    共享内存作为进程间通信的手段并重新设计了共享内存数据结构以及多写多读模式下的写入和读取机制。在服务发现机制上,
    本文选用了基于中心节点的发现模式,在中心节点中加入了网络拓扑信息保存功能,有效地降低了中心节点异常造成的通信异常。
    \item 系统的测试与验证。通过功能性测试,验证了本系统达到了需求分析中功能性需求对本系统提出的功能要求。通过
    非功能性测试中的实时性测试和稳定性测试,验证了本系统可以达到自动驾驶场景下对通信端到端延迟以及软件稳定性的性能指标。
\end{enumerate}

\section{问题和展望}
由于时间紧促,本文未能将系统的各个部分进行深入研究,有许多工作可以进一步优化或完善:
\begin{enumerate}
    \item 本系统基于共享内存的进程间通信方式序列化操作影响了通信延迟的表现,后续将开发完全传递数据指针的共享内存方案,
    减少因序列化操作带来的额外性能损失。
    \item 本系统只完成了自动驾驶对通信系统最基本的要求即通信功能,并没有完成类似ROS中的bag对整个系统
    通信过程数据的录制系统。在未来本系统需要开发给用户更多的工具链。
    \item 本系统最终的目的是运行在量产自动驾驶车辆上,而汽车行业内对于车载软件的要求是取得ASIL D安全性等级。
    本系统尚未涉及对该安全等级中提到的功能安全等方面的设计与开发。华为专注于自动驾驶操作系统内核团队约为300人,历时4年
    取得了德国莱茵ASIL D级功能安全认证,本系统在此方面任重而道远。
\end{enumerate}

综上所述,本系统目前只是完成了最基本的业务功能和性能指标,周边配套的工具链和面向量产的功能安全认证开发任务尚未涉足。

最后,本文提出的自动驾驶运行时通信系统是一个可行的方案,希望可以对自动驾驶行业有一定的帮助。


